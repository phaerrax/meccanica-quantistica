\chapter{L'oscillatore armonico}
Il sistema fisico più semplice da studiare in meccanica quantistica, contrariamente alla tradizione, non è la particella libera, in quanto l'operatore impulso (che sarebbe l'unico ad apparire nell'hamiltoniana) è un po' particolare e richiede una trattazione dedicata, che faremo in seguito.
Il sistema più semplice risulta invece essere l'oscillatore armonico, che studiamo qui in una dimensione.
Occupiamoci del problema fondamentale di individuare gli autovalori possibili dell'operatore hamiltoniano, e la loro probabilità associata.
Diamo innanzitutto l'operatore hamiltoniano, in cui sostituiamo già la costante $k$ della forza di richiamo con $m\omega^2$, in modo da evidenziare la frequenza naturale dell'oscillatore ($\omega^2=k/m$):
\begin{equation}
	\op H=\frac1{2m}(\op p^2+m^2\omega^2\op q^2).
	\label{eq:H-oscillatore}
\end{equation}
Chiamiamo $E_0,E_1,\dots,E_n,\dots$ i suoi autovalori, ordinati in modo crescente, ossia con $E_i\leq E_{i+1}$ per ogni $i$, e indichiamo di conseguenza gli autostati corrispondenti con $\ket{E_i}$.
L'autostato $\ket{E_0}$, per cui l'energia è minima, è anche chiamato \emph{autostato fondamentale}.
Cominciamo con un primo importante risultato.
\begin{teorema} \label{t:oscillatore-autovalori-positivi}
	Gli autovalori di $\op H$ di un oscillatore armonico sono tutti positivi.
\end{teorema}
\begin{proof}
	Preso uno stato qualunque $\ket{x}$, il valore di aspettazione dell'energia è
	\begin{equation}
		\avg{H}=\bra{x}\op H\ket{x}=\frac1{2m}(\bra{x}\op p^2\ket{x}+m^2\omega^2\bra{x}\op q^2\ket{x}).
	\end{equation}
	Ora, dato che $\op p$ è hermitiano, $\bra{x}\op p^2\ket{x}=\bra{x}\adj{\op p}\op p\ket{x}$ che è la norma di $\op p\ket{x}$, e lo stesso per la posizione, dunque sono entrambi positivi o nulli: allora $\bra{x}\op H\ket{x}\geq 0$ qualsiasi sia $\ket{x}$.
	Se vale per qualsiasi stato, sarà vero anche per gli autostati $\ket{E_i}$, e se li prendiamo normalizzati allora
	\begin{equation}
		0\leq\bra{E_i}\op H\ket{E_i}=E_i\braket{E_i}{E_i}=E_i
	\end{equation}
	cioè tutti gli autovalori non sono negativi.
	Se prendiamo poi l'autostato fondamentale $\ket{E_0}$, esso non può avere l'autovalore nullo: infatti $E_0=0$ se e solo se $\bra{E_0}\op H\ket{E_0}=0$, vale a dire
	\begin{equation}
		\frac1{2m}(\bra{E_0}\op p^2\ket{E_0}+m^2\omega^2\bra{E_0}\op q\ket{E_0})=0,
	\end{equation}
	ma sono tutti addendi non negativi quindi l'equazione è vera se e solo se $\bra{E_0}\op p^2\ket{E_0}=0$ e contemporaneamente $\bra{E_0}\op q^2\ket{E_0}$.
	Come prima, però, questi due sono la norma di $\op p\ket{E_0}$ e $\op q\ket{E_0}$, e ciò significherebbe che questi due vettori siano nulli, vale a dire $\op q\ket{E_0}=0$ e $\op p\ket{E_0}=0$, e di conseguenza $\ket{E_0}$ sarebbe un autostato simultaneo di $q$ e $p$.
	Questo fatto però viola il rapporto di indeterminazione per cui $\Delta q\Delta p\geq\frac{\hbar}2$ in \emph{qualsiasi} stato: in questo caso invece si avrebbe $\Delta q\Delta p=0$, in quanto sapremmo con precisione posizione e impulso (entrambi nulli).
	Allora è assurdo che $E_0$ sia nullo: se non è nemmeno positivo per quanto dimostrato precedentemente, allora $E_0>0$.
	Dato che $E_0$ è l'autovalore minimo, segue immediatamente che $E_i>0$ per ogni $i$, ossia ogni autovalore di $H$ è positivo.
\end{proof}

Nell'oscillatore armonico riconosciamo anche la simmetria che è presente anche nella controparte classica: i valori di aspettazione della posizione e dell'impulso sono zero, negli autostati di energia.
Infatti, preso un qualunque operatore $\op A$,
\begin{equation}
	\bra{E_i}[\op H,\op A]\ket{E_i}=\bra{E_i}\op H\op A\ket{E_i}-\bra{E_i}\op A\op H\ket{E_i}=E_i^*\bra{E_i}\op p\ket{E_i}-E_i\bra{E_i}\op p\ket{E_i}=0
\end{equation}
in quanto $E_i\in\R$.
Allo stesso tempo, il commutatore con $\op p$ vale
\begin{equation}
	[\op H,\op p]=\frac1{2m}[\op p^2+m^2\omega^2\op q^2,\op p]=\frac1{2m}[\op p^2,\op p]+\frac{m\omega^2}{2}[\op q^2,\op p]=\frac{m\omega^2}2(\op q[\op q,\op p]+[\op q,\op p]\op q)=i\hbar m\omega^2\op q\ne 0
	\label{eq:commutatore-Hp}
\end{equation}
cioè $\op q=[\op H,\op p]/i\hbar m\omega^2$.
Di conseguenza il valore di aspettazione di $q$ da un autostato di energia risulta
\begin{equation}
	\bra{E_i}\op q\ket{E_i}=\frac1{i\hbar m\omega^2}\bra{E_i}[\op H,\op p]\ket{E_i}=0.
\end{equation}
Con lo stesso ragionamento, calcolando $[\op H,\op q]$ si giunge a $\bra{E_i}\op p\ket{E_i}=0$.

Calcoliamo infine lo stato di energia minima: assumendo $\braket{E_0}{E_0}=1$, abbiamo
\begin{equation}
	E_0=\bra{E_0}\op H\ket{E_0}=\frac1{2m}(\bra{E_0}\op p^2\ket{E_0}+m^2\omega^2\bra{E_0}\op q^2\ket{E_0}),
\end{equation}
ma $\bra{E_0}\op p^2\ket{E_0}=\avg{p^2}$, e dato che $\avg{p}=0$ si ha $\Delta p^2=\avg{p^2}-\avg{p}^2=\avg{p^2}$ (analogamente per $q$), quindi\footnote{Sfruttiamo la disuguaglianza $a^2+b^2\ge 2ab$, per $a,b\in\R$: basta considerare $(a-b)^2\ge 0$ per dimostrarla.}
\begin{equation}
	E_0=\frac1{2m}(\avg{p^2}+m^2\omega^2\avg{q^2})=\frac1{2m}(\Delta p^2+m^2\omega^2\Delta q^2)\ge \frac1{2m}2m\omega\Delta p\Delta q=\omega\Delta p\Delta q\ge\frac{\hbar\omega}2,
\end{equation}
e in particolare vale proprio $\hbar\omega/2$ soltanto se $\Delta p=m\omega\Delta q$.
Pertanto lo stato di minima energia, che è allo stesso tempo lo stato di \emph{minima indeterminazione}, è dato da
\begin{equation}
	\begin{cases}
		\Delta p=m\omega\Delta q\\ \Delta p\Delta q=\frac{\hbar}2.
	\end{cases}
	\label{eq:oscillatore-armonico-stato-minima-energia}
\end{equation}
