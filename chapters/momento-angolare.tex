\chapter{Momento angolare}
L'operatore associato al momento angolare è, come in meccanica classica, il generatore delle rotazioni, che soddisfa le relazioni di commutazione
\begin{equation}
	[\op L_i,\op L_j]=i\hbar\epsilon_{ijk}\op L_k.
	\label{eq:commutazione-momento-angolare}
\end{equation}
A meno di fattori moltiplicativi, le costanti di struttura (la parte rilevante è il simbolo di Levi-Civita $\epsilon_{ijk}$) sono le stesse dell'algebra di Lie $\mathfrak{su}(2)$.
Non a caso, $\mathfrak{su}(2)$ è isomorfa a $\mathfrak{so}(3)$ che è l'algebra del gruppo $SO(3)$ delle rotazioni tridimensionali; in altre parole $\mathfrak{su}(2)$ genera $SO(3)$, allo stesso modo in cui il momento angolare genera le rotazioni in $\R^3$, e ciò giustifica la \eqref{eq:commutazione-momento-angolare}.
Una quantità che commuta con il momento angolare lungo una data direzione è dunque simmetrica per le rotazioni attorno a tale asse: gli scalari sono degli importanti esempi di queste quantità.
\footnote{
	Spesso si usa, per alleggerire la notazione, porre $\hbar=1$ per eliminarlo sistematicamente dalle equazioni.
	In tal caso si suppone di misurare il momento angolare (e le osservabili che ne derivano) in unità di $\hbar$.
	In pratica, il punto di partenza è dividere la \eqref{eq:commutazione-momento-angolare} per $\hbar^2$ ottenendo la forma più semplice $[\op L_i,\op L_j]=i\epsilon_{ijk}\op L_k$.
}
Insieme alle tre componenti cartesiane del momento angolare abbiamo anche il quadrato del suo modulo, ossia $\op{\vec L}^2=\op L_i\op L_i$: essendo uno scalare, ci aspettiamo che sia invariante per rotazioni ossia che commuti con tutte le componenti del momento angolare.
Una verifica diretta mostra infatti che, per $i=1,2,3$,
\begin{equation}
	[\op{\vec L}^2,\op L_i]=[\op L_k\op L_k,\op L_i]=\op L_k[\op L_k,\op L_i]+[\op L_k,\op L_i]\op L_k=i\hbar\epsilon_{kil}(\op L_k\op L_l+\op L_l\op L_k)=i\hbar(\epsilon_{kil}+\epsilon_{lik})\op L_k\op L_l=0
\end{equation}
per l'antisimmetria di $\epsilon_{kil}$.
In ogni caso le tre componenti $\op L_i$ non sono tra loro compatibili (dalla \eqref{eq:commutazione-momento-angolare}) a meno del caso banale in cui siano tutte nulle.
Nella costruzione di un sistema completo di osservabili compatibili, è convenzione scegliere insieme a $\op{\vec L}^2$ anche la componente $z$, ossia $\op L_3$, da diagonalizzare simultaneamente.

\section{Autovalori del momento angolare} \label{sec:autovalori-momento-angolare}
Supponiamo di aver trovato un sistema completo di osservabili compatibili che contenga $\op{\vec L}^2$ e $\op L^3$.
Chiamiamo $m\hbar$ gli autovalori di $\op L_3$ e $\mu^2\hbar^2$ quelli di $\op{\vec L}^2$, e trascurando lo altre osservabili del sistema indichiamo gli autostati simultanei con $\ket{m,\mu^2}$, che sono quindi tali per cui\footnote{Gli autovalori di $\op{\vec L}^2$ sono evidentemente positivi, perciò li indichiamo direttamente con il quadrato $\mu^2$.}
\begin{equation}
	\op{\vec L}^2\ket{m,\mu^2}=\mu^2\hbar^2\ket{m,\mu^2}
	\hspace{1cm}\text{e}\hspace{1cm}
	\op L_3\ket{m,\mu^2}=m\hbar\ket{m,\mu^2}.
\end{equation}
Come già per l'oscillatore armonico, introduciamo gli operatori a scala $\op L_+\defeq\op L_1+i\op L_2$ e $\op L_-\defeq\op L_1-i\op L_2$.
Dato che gli $\op L_i$ sono hermitiani (poich\'e rappresentano delle osservabili) si vede subito che $\adj{\op L_+}=\op L_-$.
Essi sono in relazione con la componente scelta del momento angolare con i commutatori
\begin{equation}
	[\op L_3,\op L_+]=[\op L_3,\op L_1]+i[\op L_3,\op L_2]=\hbar(i\op L_2-i^2\op L_1)=\hbar\op L_+
\end{equation}
da cui
\begin{equation}
	[\op L_3,\op L_-]=[\op L_3,\adj{\op L_+}]=-\adj{[\op L_3,\op L_+]}=-\hbar\adj{\op L_+}=-\hbar\op L_-.
\end{equation}
Vediamo dunque come questi due operatori agiscono sugli autostati:
\begin{multline}
	\op L_3\op L_+\ket{m,\mu^2}=
	(\op L_3\op L_+ - \op L_+\op L_3 + \op L_+\op L_3)\ket{m,\mu^2}=
	[\op L_3,\op L_+]\ket{m,\mu^2}+\op L_+\op L_3\ket{m,\mu^2}=\\=
	\hbar\op L_+\ket{m,\mu^2}+m\hbar\op L_+\ket{m,\mu^2}=
	(m+1)\hbar\op L_+\ket{m,\mu^2}
\end{multline}
e analogamente $\op L_3\op L_-\ket{m,\mu^2}=(m-1)\hbar\op L_-\ket{m,\mu^2}$.
Gli operatori $\op L_+$ e $\op L_-$ dunque portano un autostato di $\op L_3$ in un altro autostato con autovalore, rispettivamente, aumentato o diminuito di $\hbar$.
Più brevemente scriviamo che $\op L_+\ket{m,\mu^2}=\ket{m+1,\mu^2}$ e $\op L_-\ket{m,\mu^2}=\ket{m-1,\mu^2}$.

Dobbiamo quindi chiederci se la sequenza $\{\dots,m-2,m-1,m,m+1,\dots\}$ sia limitata o meno.
Dato che $\op L_3^2=\op{\vec L}^2-\op L_1^2+\op L_2^2$, sicuramente per ciascuno stato il valore di aspettazione di $\op L_3^2$ deve essere minore di quello di $\op{\vec L}^2$, dato che ovviamente $\op L_1^2+\op L_2^2$ è un operatore definito positivo.
In particolare, dunque, risulta $m^2\le\mu^2$, ossia $\abs{m}\le\abs{\mu}$: gli autovalori sono allora necessariamente limitati; esiste un certo stato che chiamiamo $\ket{j_+,\mu^2}$ tale per cui $\op L_+\ket{j_+,\mu^2}=0$ e analogamente uno stato $\ket{j_-,\mu^2}$ per cui $\op L_-\ket{j_-,\mu^2}=0$, dove $j_+$ e $j_-$ sono il massimo e il minimo autovalore, rispettivamente, di $\op L_3$.
Per questi due stati si ha comunque $\op{\vec L}^2\ket{j_\pm,\mu^2}=\mu^2\hbar^2\ket{j_\pm,\mu^2}$.
Calcolando la norma di $\op L_+\ket{j_+,\mu^2}$ troviamo
\begin{equation}
	\begin{split}
		0&=\bra{j_+,\mu^2}\adj{\op L_+}\op L_+\ket{j_+,\mu^2}=\\
		&=\bra{j_+,\mu^2}\op L_-\op L_+\ket{j_+,\mu^2}=\\
		&=\bra{j_+,\mu^2}(\op L_1-i\op L_2)(\op L_1+i\op L_2)\ket{j_+,\mu^2}=\\
		&=\bra{j_+,\mu^2}(\op L_1^2+\op L_2^2+i[\op L_1,\op L_2])\ket{j_+,\mu^2}=\\
		&=\bra{j_+,\mu^2}(\op{\vec L}^2-\op L_3^2-\hbar\op L_3)\ket{j_+,\mu^2}=\\
		&=(\mu^2-j_+^2-j_+)\hbar^2\braket{j_+,\mu^2}{j_+,\mu^2}
	\end{split}
\end{equation}
da cui otteniamo $\mu^2=j_+(j_++1)$.
Dalla norma di $\op L_-\ket{j_-,\mu^2}$, anch'essa nulla, troviamo invece $\mu^2=j_-(j_--1)$.
Assegnato un valore a $j_-$, uguagliando le due espressioni abbiamo $j_+(j_++1)=j_-(j_--1)$ da cui $j_+=-j_-$ oppure $j_+=j_--1$.
La seconda delle due però non è accettabile dato che per costruzione $j_+\ge j_-$, perciò rimane $j_+=-j_-$.
Chiamiamo questo valore semplicemente con $j$.
Abbiamo quindi ottenuto lo spettro di $\op L_3$ che è l'insieme $\{-j\hbar,(-j+1)\hbar,\dots,(j-1)\hbar,j\hbar\}$, che ha $2j+1$ elementi.
Questa cardinalità è chiaramente un numero intero, perciò $j$ deve essere a sua volta intero oppure semiintero (positivo o nullo).
Una volta che $\op L_3$ assume uno di questi valori, poi, $\op{\vec L}^2$ ha come autovalore $j(j+1)\hbar^2$: tale numero non è un quadrato perfetto, per il fatto che anche quando $j=m$ (per il massimo autovalore di $\op L_3$) non si ha comunque $\mu^2=m^2$, perch\'e da questo seguirebbe che $\op L_1$ e $\op L_2$ avrebbero solo autovalori nulli, ossia il momento angolare lungo i due assi restanti sarebbe nullo; sappiamo che questo non è possibile, perch\'e non possiamo determinare con precisione assoluta contemporaneamente due componenti del momento angolare.
L'unica eccezione a questo si ha nel caso banale in cui $j=0$, in cui $\op{\vec L}=0$.

In questa rappresentazione del momento angolare, inoltre, $\op{\vec L}^2$ è un multiplo dell'identità (detto anche \emph{operatore di Casimir}): è degenere dato che ad ogni suo autostato corrispondono $2j+1$ autostati linearmente indipendenti (di $\op L_3$), come si poteva anche capire dal teorema \ref{t:degenerazione} oppure dal lemma di Schur.
La rappresentazione è inoltre determinata completamente determinata dal numero $j$.\footnote{Ogni rappresentazione del gruppo delle rotazioni è univocamente determinata da un numero intero o semiintero positivo o nullo. Tutte le rappresentazioni irriducibili si ottengono in questo modo, mentre quelle riducibili si ricavano come somma diretta o prodotto di queste.}
Ad esempio la rappresentazione con $j=\frac12$ è data dalle matrici di Pauli: si ha infatti $\sigma_3=\frac12\begin{psmallmatrix}1&0\\0&-1\end{psmallmatrix}$ e $\frac14(\sigma_1^2+\sigma_2^2+\sigma_3^2)=\frac34=j(j+1)$ che corrisponde a $\op{\vec L}^2$.

\section{Momento angolare orbitale}
Possiamo definire l'operatore del momento angolare \emph{orbitale} di una particella attorno all'origine, in coordinate cartesiane, utilizzando la definizione classica e sostituendo gli operatori corrispondenti: otteniamo
\begin{equation}
	\op L_k=\epsilon_{ijk}\op x_i\op p_j
	\label{eq:momento-angolare-orbitale}
\end{equation}
in dimensione 3.
Si può eventualmente generalizzare a un numero differente di dimensioni seguendo le regole di commutazione dell'algebra $\mathfrak{so}(n)$.
Nella rappresentazione di Schr\"odinger della posizione troviamo che $\op L_i$ è dato da
\begin{equation}
	L_i=\epsilon_{ijk}x_j\bigg(-i\hbar\drp{}{x_k}\bigg)=-i\hbar\epsilon_{ijk}x_j\drp{}{x_k}.
\end{equation}
Possiamo ottenere però una rappresentazione più vantaggiosa usando le coordinate polari, più adatte a descrivere ad esempio sistemi a simmetria centrale, cioè invarianti per rotazioni.
Usiamo la trasformazione
\begin{equation}
	\begin{pmatrix}
		x_1\\
		x_2\\
		x_3
	\end{pmatrix}
	=
	\begin{pmatrix}
		r\sin\theta\cos\phi\\
		r\sin\theta\sin\phi\\
		r\cos\theta
	\end{pmatrix}:
\end{equation}
le tre componenti del momento angolare orbitale, in rappresentazione di Schr\"odinger negli autostati della posizione, sono
\begin{equation}
	\begin{split}
		L_1=&-i\hbar\bigg(x_2\drp{}{x_3}-x_3\drp{}{x_2}\bigg)=\\
		=&-i\hbar\bigg[x_2\bigg(\drp{r}{x_3}\drp{}{r}+\drp{\theta}{x_3}\drp{}{\theta}+\drp{\phi}{x_3}\drp{}{\phi}\bigg)
			-x_3\bigg(\drp{r}{x_2}\drp{}{r}+\drp{\theta}{x_2}\drp{}{\theta}+\drp{\phi}{x_2}\drp{}{\phi}\bigg)\bigg]=\\
		=&-i\hbar\bigg[r\sin\theta\sin\phi\bigg(\!\!\cos\theta\drp{}{r}-\frac{\sin\theta}{r}\drp{}{\theta}\bigg)
			-r\cos\theta\bigg(\!\!\sin\theta\sin\phi\drp{}{r}+\frac{\cos\theta\sin\phi}{r}\drp{}{\theta}+\frac{\cos\phi}{r\sin\theta}\drp{}{\phi}\bigg)\bigg]=\\
		=&-i\hbar\bigg(r\cos\theta\sin\theta\sin\phi\drp{}{r}-\sin^2\theta\sin\phi\drp{}{\theta}-r\cos\theta\sin\theta\sin\phi\drp{}{r}+\\
			&+\cos^2\theta\sin\phi\drp{}{\theta}-\cot\theta\cos\phi\drp{}{\phi}\bigg)=\\
		=&\,i\hbar\bigg(\!\sin\phi\drp{}{\theta}+\cot\theta\cos\phi\drp{}{\phi}\bigg);
	\end{split}
	\label{eq:L_1-schroedinger}
\end{equation}
\begin{equation}
	\begin{split}
		L_2=&-i\hbar\bigg(x_3\drp{}{x_1}-x_1\drp{}{x_3}\bigg)=\\
		=&-i\hbar\bigg(r\cos\theta\sin\theta\cos\phi\drp{}{r}-\cot\theta\sin\phi\drp{}{\phi}+\cos^2\theta\cos\phi\drp{}{\theta}-r\cos\theta\sin\theta\cos\phi\drp{}{r}+\\
			&+\sin^2\theta\cos\phi\drp{}{\theta}\bigg)=\\
		=&-i\hbar\bigg(\!\cos\phi\drp{}{\theta}-\cot\theta\sin\phi\drp{}{\phi}\bigg);
	\end{split}
	\label{eq:L_2-schroedinger}
\end{equation}
\begin{equation}
	\begin{split}
		L_3=&-i\hbar\bigg(x_1\drp{}{x_2}-x_2\drp{}{x_1}\bigg)=\\
		=&-i\hbar\bigg[r\cos\phi\sin\phi\sin^2\theta\drp{}{r}+\cos\theta\sin\theta\cos\phi\sin\phi\drp{}{\theta}+\cos^2\phi\drp{}{\phi}-r\cos\phi\sin\phi\sin^2\theta\drp{}{r}-\\
			&-\cos\theta\sin\theta\cos\phi\sin\phi\drp{}{\theta}+\sin^2\theta\drp{}{\phi}\bigg]=\\
		=&-i\hbar\drp{}{\phi}.
	\end{split}
	\label{eq:L_3-schroedinger}
\end{equation}
Da questi otteniamo gli operatori a scala
\begin{equation}
	\begin{split}
		L_+&=L_1+iL_2=i\hbar\bigg(\!\sin\phi\drp{}{\theta}+\cot\theta\cos\phi\drp{}{\phi}\bigg)+\hbar\bigg(\!\cos\phi\drp{}{\theta}-\cot\theta\sin\phi\drp{}{\phi}\bigg)=\\
		&=\hbar\bigg[(i\sin\phi+\cos\phi)\drp{}{\theta}+(i\cos\phi-\sin\phi)\cot\theta\drp{}{\phi}\bigg]=\\
		&=\hbar e^{i\phi}\bigg(\drp{}{\theta}+i\cot\theta\drp{}{\phi}\bigg)
	\end{split}
	\label{eq:L_+-schroedinger}
\end{equation}
e in modo analogo
\begin{equation}
	L_-=L_1-iL_2=\hbar e^{-i\phi}\bigg(\!-\drp{}{\theta}+i\cot\theta\drp{}{\phi}\bigg).
	\label{eq:L_--schroedinger}
\end{equation}
Con queste espressioni possiamo infine rappresentare anche $\op{\vec L}^2$, ricordando che è uguale a $\op L_+\op L_-+\op L_3^2-\hbar\op L_3$: siccome
\begin{equation}
	L_+L_-=-\hbar^2\bigg(\ddrp{}{\theta}+\cot\theta\drp{}{\theta}+\cot^2\theta\ddrp{}{\phi}+i\drp{}{\phi}\bigg)
\end{equation}
troviamo che
\begin{equation}
	\begin{split}
		\vec L^2&=-\hbar^2\bigg(\ddrp{}{\theta}+\cot\theta\drp{}{\theta}+\cot^2\theta\ddrp{}{\phi}+i\drp{}{\phi}+\ddrp{}{\phi}-i\drp{}{\phi}\bigg)=\\
		&=-\hbar^2\bigg(\ddrp{}{\theta}+\cot\theta\drp{}{\theta}+\frac1{\sin^2\theta}\ddrp{}{\phi}\bigg)=\\
		&=-\hbar^2\bigg[\frac1{\sin\theta}\bigg(\!\sin\theta\ddrp{}{\theta}+\cos\theta\drp{}{\theta}\bigg)+\frac1{\sin^2\theta}\ddrp{}{\phi}\bigg]=\\
		&=-\hbar^2\bigg[\frac1{\sin\theta}\drp{}{\theta}\bigg(\!\sin\drp{}{\theta}\bigg)+\frac1{\sin^2\theta}\ddrp{}{\phi}\bigg]
	\end{split}
	\label{eq:L^2-schroedinger}
\end{equation}
e il termine tra le parentesi quadre, a meno del fattore $1/r^2$, è la parte angolare del laplaciano in coordinate sferiche.

Affinch\'e la funzione d'onda sia continua, deve risultare periodica di $2\pi$ nella variabile $\phi$.
Cos\`i come una funzione definita su un intervallo limitato si può analizzare in serie di Fourier e ammette dei modi normali di oscillazione con un insieme discreto (numerabile) di frequenze, anche la parte in $\phi$ delle autofunzioni di $\op L_3$ avrà questa caratteristica: questo porta, tra le altre cose, alla quantizzazione del momento angolare.
Le autofunzioni di questo operatore sono infatti funzioni $f(r,\phi,\theta)$ tali che
\begin{equation}
	-i\hbar\drp{f}{\phi}(r,\phi,\theta)=m\hbar f(r,\phi,\theta)\qqq f(r,\phi,\theta)=g(r,\theta)e^{im\phi}.
\end{equation}
Imponendo la periodicità della funzione otteniamo che deve soddisfare
\begin{equation}
	g(r,\theta)e^{im\phi}=f(r,\theta,\phi+2\pi)=g(r,\theta)e^{im(\phi+2\pi)}=g(r,\theta)e^{im\phi}e^{2im\pi}
\end{equation}
da cui ricaviamo che $m$ deve essere intero in modo che $e^{2im\pi}=1$.


Rimane da rappresentare anche $\op{\vec L}^2$: dall'identità, valida in una generica dimensione $d$,
\begin{equation}
	\op{\vec L}^2=\op{\vec x}^2\op{\vec p}^2-(\scalar{\op{\vec x}}{\op{\vec p}})^2+i\hbar(d-2)\scalar{\op{\vec x}}{\op{\vec p}}
	\label{eq:relazione-L2-x-p}
\end{equation}
possiamo anche ottenere una comoda rappresentazione di $\op{\vec p}^2$ in coordinate polari: è sufficiente moltiplicare a sinistra per l'operatore $\op{\vec x}^{-2}$ (l'inverso del quadrato della norma).
In pratica, la norma al quadrato dell'operatore di posizione corrisponde al quadrato della distanza dall'origine, cioè $r^2$, che sarà l'autovalore di questo operatore in una rappresentazione in cui è diagonale.

Un'operatore hamiltoniano della forma $\frac1{2m}\op{\vec p}^2+V(\norm{\op{\vec x}}^2)$ è invariante per rotazioni e commuta dunque con $\op{\vec L}^2$ e $\op L_3$.
In generale sarà un operatore degenere: possiamo etichettare gli autostati con tre numeri $E,l,m$ tali che, oltre all'equazione di Schr\"odinger, abbiamo
\begin{equation}
	L^2\psi_{E,l,m}(\vec x)=l(l+1)\hbar^2\psi_{E,l,m}(\vec x)\qeq L_3\psi_{E,l,m}(\vec x)=m\hbar\psi_{E,l,m}(\vec x).
	\label{eq:autofunzioni-momento-angolare}
\end{equation}
Vediamo dunque come rappresentare opportunamente $\op{\vec p}^2$ in $L^2(\R^3)$ con l'identità \eqref{eq:relazione-L2-x-p}: nella rappresentazione delle coordinate $\op{\vec x}^2$ è diagonale si traduce nella moltiplicazione per $r^2$ (analogamente per $\op{\vec x}^{-2}$), mentre
\begin{equation}
	\scalar{\op{\vec x}}{\op{\vec p}}=-i\hbar x_i\drp{}{x_i}=-i\hbar r\frac{x_i}{r}\drp{}{x_i}=-i\hbar r\drp{r}{x_i}\drp{}{x_i}=-i\hbar r\drp{}{r}.
\end{equation}
Analogamente alla coppia di coordinate cartesiane, possiamo definire un operatore $\op r\defeq\sqrt{\op x^2+\op y^2+\op z^2}$ (in senso generalizzato: abbiamo gli stessi problemi di $x$ e $p$), che risulta autoaggiunto, e un momento coniugato $\op p_r\defeq-i\hbar\drp{}{r}$.
Quest'ultimo però non è autoaggiunto: se per verificarlo nel casi dei $\op p_i$ cartesiani era sufficiente integrare per parti dato che i termini al contorno (per $\abs{x}\to+\infty$) erano nulli, in questo caso il contorno del dominio di $r$ è $0$ e $+\infty$, e non c'è alcun motivo di supporre che le funzioni d'onda siano nulle nell'origine; oltretutto, la misura in coordinate polari non è il semplice prodotto dei $\dd x_i$ ma contiene altri fattori.
Con queste definizioni abbiamo dunque
\begin{equation}
	(\scalar{\op{\vec x}}{\op{\vec p}})^2=-\hbar^2r\drp{}{r}r\drp{}{r}=-\hbar^2\bigg(r\drp{}{r}+r^2\ddrp{}{r}\bigg)=\op r^2\op p_r^2-i\hbar \op r\op p_r
\end{equation}
e il quadrato dell'impulso si scrive finalmente come
\begin{equation}
	\op{\vec p}^2=-\hbar^2\ddrp{}{r}-\hbar^2(d-1)\frac1{r}\drp{}{r}+\frac{L^2}{r^2}=
	-\hbar^2\bigg[\ddrp{}{r}+(d-1)\frac1{r}\drp{}{r}\bigg]+\frac{L^2}{r^2}=
	-\hbar^2\lap
\end{equation}
dato che
\begin{equation}
	\lap=\ddrp{}{r}+(d-1)\frac1{r}\drp{}{r}-\frac{L^2}{\hbar^2r^2}=
	\frac1{r^{d-1}}\drp{}{r}r^{d-1}\drp{}{r}-\frac{L^2}{\hbar^2r^2}
	\label{eq:laplaciano-momento-angolare}
\end{equation}
è l'operatore laplaciano, nella nostra rappresentazione in coordinate sferiche in dimensione $d$.

Notiamo che nelle equazioni agli autovalori \eqref{eq:autofunzioni-momento-angolare} per le funzioni $\psi_{E,l,m}$ non sono coinvolti il potenziale e l'energia del sistema, ma solo i numeri $l$ e $m$.
Per cercare le autofunzioni scegliamo innanzitutto un potenziale nullo, ottenendo l'equazione $-\hbar^2\lap\psi=2mE\psi$.
Ponendo $k^2\defeq\frac{2mE}{\hbar^2}$ possiamo semplificarla nella forma $\lap\psi=-k^2\psi$.
Effettuiamo dunque una separazione delle variabili fattorizzando la funzione d'onda come $R(r)Y(\phi,\theta)$: nella prima delle \eqref{eq:autofunzioni-momento-angolare} troviamo cos\`i $L^2R(r)Y(\phi,\theta)=\hbar^2l(l+1)R(r)Y(\phi,\theta)$ in cui semplifichiamo $R(r)$ trovando
\begin{equation}
	L^2Y(\phi,\theta)=\hbar^2l(l+1)Y(\phi,\theta)
\end{equation}
e con un ragionamento analogo, detto $g(l)$ l'autovalore di $L^2$ in dimensione generica $d$, l'equazione $\lap\psi=-k^2\psi$ si riscrive con la \eqref{eq:laplaciano-momento-angolare} come
\begin{equation}
	Y(\phi,\theta)\bigg[\frac1{r^{d-1}}\drp{}{r}r^{d-1}\drp{}{r}+\frac{g(l)}{r^2}\bigg]R(r)=-k^2R(r)Y(\phi,\theta)
	\label{eq:laplace-parte-radiale}
\end{equation}
e possiamo semplificare il fattore $Y(\phi,\theta)$ dai due membri.
Ora possiamo porre anche $E=0$, ottenendo per la funzione d'onda l'equazione di Laplace $\lap\psi=0$: essa è tale che se $\psi(\vec x)$ è soluzione, lo è anche $\psi(\lambda\vec x)$ per ogni $\lambda\in\R$, ossia le soluzioni sono funzioni omogenee.
Cercheremo le soluzioni a questa equazione, quindi, nell'insieme dei polinomi omogenei.

\paragraph{Equazione di Laplace in $\R^2$}
Data l'equazione $\lap u(x,y)=0$, sfruttiamo l'isomorfismo tra $\R^2$ e $\C$ per semplificare le operazioni, con il cambio di coordinate
\begin{equation}
	x=\frac{\eta+\eta^*}2\qeq y=\frac{\eta-\eta^*}{2i}
\end{equation}
ossia ponendo $\eta\defeq x+iy$.
L'operatore laplaciano si riscrive come
\begin{equation}
	\ddrp{}{x}+\ddrp{}{y}=4\frac{\partial^2}{\partial\eta\partial\eta^*}
\end{equation}
per cui l'equazione di Laplace diventa $\frac{\partial^2}{\partial\eta\partial\eta^*}u(\eta,\eta^*)=0$.
Se ora $u$ è una funzione olomorfa, non dipende da $\eta^*$ quindi è evidentemente soluzione dell'equazione; analogamente se è antiolomorfa, ossia non dipende da $\eta$.
L'equazione $\lap u=0$ è inoltre invariante per trasformazioni conformi, ossia mappe $\C\to\C$ olomorfe.
In ogni caso, prendiamo un generico polinomio in $\eta$ e $\eta^*$ omogeneo di grado $l$
\begin{equation}
	u_l(\eta,\eta^*)=\sum_{p=0}^lc_p\eta^p(\eta^*)^{l-p}:
\end{equation}
affinch\'e sia soluzione dell'equazione di Laplace deve risultare
\begin{equation}
	0=\frac{\partial^2}{\partial\eta\partial\eta^*}\sum_{p=0}^lc_p\eta^p(\eta^*)^{p-l}=\sum_{p=0}^lc_p(l-p)p\eta^{p-1}(\eta^*)^{l-p-1}
\end{equation}
ossia $p(l-p)c_p=0$ per ogni $p=0,\dots,l$, a meno che $p=l$ o $l=0$.
Allora rimane soltanto $u(\eta,\eta^*)=c_0(\eta^*)^l+c_l\eta^l$, e tornando alle variabili reali abbiamo dunque
\begin{equation}
	\begin{gathered}
		c_0(\eta^*)^l=c_0(x-iy)^l=c_0r^l(\cos\phi-i\sin\phi)^l=c_0r^le^{-il\phi},\\
		c_l\eta^l=c_l(x+iy)^l=c_lr^l(\cos\phi+i\sin\phi)^l=c_lr^le^{il\phi}.
	\end{gathered}
\end{equation}

Tornando al problema generale, in una dimensione generica, alla luce di questi risultati è sensato carcare delle soluzioni, in coordinate polari, $u(r,\phi_1,\dots,\phi_{n-1})$ che siano fattorizzate come $r^lY(\phi_1,\dots,\phi_{n-1})$.
Inoltre dalla \eqref{eq:laplace-parte-radiale}, dopo aver semplificato la funzione $Y$, abbiamo
\begin{equation}
	0=-\frac1{r^{d-1}}\drp{}{r}r^{d-1}\drp{}{r}r^l=-l\frac1{r^{d-1}}\drp{}{r}r^{d-1+l-1}=-l(d+l-2)\frac1{r^{d-1}}r^{d-l-2}=-l(d+l-2)r^{l-2}
\end{equation}
perciò la funzione che dà gli autovalori di $L^2$ in dimensione $d$ è $g(l)=l(d+l-2)$.
In particolare, per $d=2$ troviamo $g(l)=l^2$, e infatti il generatore di $L^2$ è unico in $\R^2$ (c'è solo una componente del momento angolare), e tale operatore ha come autovalori $\pm l$, a cui corrispondono le autofunzioni $c_0r^le^{-il\phi}$ e $c_lr^le^{il\phi}$ trovate in precedenza.

\paragraph{Equazione di Laplace in $\R^3$}
Per risolvere l'equazione $\lap u(x,y,z)=0$ effettuiamo la stessa sostituzione nel caso bidimensionale sulle coordinate $x,y$ lasciando invariata la terza.
Un polinomio omogeneo di grado $l$ nelle variabili $\eta,\eta^*,z$ è della forma
\begin{equation}
	u_l(\eta,\eta^*,z)=\sum_{p,q=0}^lc_{p,q}\eta^p(\eta^*)^qz^{l-p-q}
\end{equation}
mentre l'operatore laplaciano si estende facilmente dal caso precedente come $\lap=4\frac{\partial}{\partial\eta\partial\eta^*}+\frac{\partial^2}{\partial z^2}$: il polinomio $u_l$ è soluzione dell'equazione di Laplace, dunque, se
\begin{equation}
	\begin{split}
		0&=\sum_{p,q=0}^l4c_{p,q}\frac{\partial^2}{\partial\eta\partial\eta^*}\eta^p(\eta^*)^qz^{l-p-q}+\sum_{p,q=0}^lc_{p,q}\ddrp{}{z}\eta^p(\eta^*)^qz^{l-p-q}=\\
		&=\sum_{p,q=0}^l4c_{p,q}pq\eta^{p-1}(\eta^*)^{q-1}z^{l-p-q}+\sum_{p,q=0}^lc_{p,q}(l-p-q)(l-p-q-1)\eta^p(\eta^*)^qz^{l-p-q-2}
	\end{split}
\end{equation}
e riscalando gli indici della prima somma risulta
\begin{equation}
	\sum_{p,q=-1}^{l-1}4c_{p+1,q+1}(p+1)(q+1)\eta^{p}(\eta^*)^{q}z^{l-p-q-2}=-\sum_{p,q=0}^lc_{p,q}(l-p-q)(l-p-q-1)\eta^p(\eta^*)^qz^{l-p-q-2}
\end{equation}
da cui otteniamo l'equazione di riccorrenza
\begin{equation}
	c_{p+1,q+1}=-\frac{(l-p-q)(l-p-q-2)}{4(p+1)(q+1)}c_{p,q}.
	\label{eq:ricorrenza-coefficienti-equazione-laplace}
\end{equation}
Possiamo disporre i coefficienti $c_{p,q}$ su un piano, in cui risulteranno limitati dagli assi $p=0$, $q=0$ e dalla retta $p+q=l$.
Prendendo uno dei punti nel grafico, il termine successivo nella ricorrenza si ottiene aumentando di 1 entrambi gli indici, muovendosi dunque in diagonale nel grafico fino a raggiungere il bordo.
I coefficienti sui due assi possono essere di conseguenza assunti come condizioni iniziali per le soluzioni: tutti gli altri si ottengono da questi e dalla \eqref{eq:ricorrenza-coefficienti-equazione-laplace}.
Queste condizioni iniziali sono in tutti $2l+1$, che è quindi il numero di soluzioni indipendenti dell'equazione di Laplace.
Indichiamo dunque con i due numeri $l$ e $m$, dove $m$ assume uno di questi $2l+1$ valori, le varie soluzioni indipendenti dell'equazione differenziale, che saranno quindi della forma $r^lY_{l,m}(\phi,\theta)$.

\paragraph{Armoniche sferiche}
Le funzioni $Y_{l,m}$ sono dette \emph{armoniche sferiche}.
Indichiamo con $Y_{l,m}$ la soluzione che ha come condizione iniziale $c_{m,0}$ e con $Y_{l,-m}$ la soluzione con condizione iniziale $c_{0,m}$.
La funzione $Y_{0,0}$ è costante, e corrisponde alla rappresentazione del gruppo delle rotazioni di peso $l=0$.
Affinch\'e le armoniche sferiche siano normalizzate, devono essere integrate sull'angolo solido di $4\pi$, perciò ad esempio $Y_{0,0}=\frac1{\sqrt{4\pi}}$.
Guardiamo invece le armoniche sferiche con $l=1$ abbiamo
\begin{equation}
	\begin{aligned}
		rY_{1,-1}=r\sin\theta e^{-i\phi}=x-iy\\
		rY_{1,0}=r\cos\theta=z\\
		rY_{1,1}=r\sin\theta e^{-i\phi}=x+iy
	\end{aligned}
	\label{eq:armoniche-sferiche-1}
\end{equation}
che sono combinazioni lineari (complesse) delle tre componenti del vettore posizione: queste tre funzioni si trasformano dunque come vettori sotto l'effetto di una rotazione.
Prendiamo ora l'armonica $Y_{l,l}$: troviamo che $r^lY_{l,l}=\eta^l=(x+iy)^l=r^l\sin^l\theta e^{il\phi}$ che è dunque anche autofunzione di $L_3$.
Analogamente, lo è anche $r^lY_{l,-l}=(\eta^*)^l=r^l\sin^l\theta e^{-il\phi}$.
In generale, risulta
\begin{equation}
	r^lY_{l,m}(\phi,\theta)=\sum_kc_{m+k,k}\eta^{m+k}(\eta^*)^kz^{l-m-2k}=r^l\sum_kc_{m+k,k}\sin^{m+2k}\theta\cos^{l-m-2k}\theta e^{im\phi}
\end{equation}
e si nota subito che è un'autofunzione di $L_3$ con autovalore $m$.
Se poi $m$ è pari il termine $\sin\theta$ è elevato ad una potenza pari, dunque si ha una funzione di $\sin^2\theta=1-\cos^2\theta$: la funzione $r^lY_{l,m}(\phi,\theta)$, nella variabile $\theta$, è allora un polinomio in $\cos\theta$.
Una volta normalizzato, risulta
\begin{equation}
	Y_{l,m}(\phi,\theta)=\sqrt{\frac{2l+1}{4\pi}\frac{(l-\abs{m})!}{(l+\abs{m})!}}e^{im\phi}P_l^{\abs{m}}(\cos\theta)
	\label{eq:armoniche-sferiche-funzioni-associate-legendre}
\end{equation}
indicando con $P_l^{\abs{m}}$ è la \emph{funzione associata di Legendre} (per $m$ pari), definita come
\begin{equation}
	P_l^k(x)=(-1)^k(1-x^2)^{\frac{k}2}\frac{\dd^k}{\dd x^k}P_l(x)
	\label{eq:funzione-associata-legendre}
\end{equation}
dove $P_l(x)$ è il \emph{polinomio di Legendre} di grado $l$: si vede che $P_l^k$ è un polinomio solo se $k$ è pari, altrimenti compaiono delle radici nella sua espressione.
I polinomi di Legendre formano un insieme ortogonale nel dominio $[-1,1]$: risulta
\begin{equation}
	\begin{gathered}
		\int_{-1}^1P_l(x)P_k(x)\,\dd x=\frac{2}{2l+1}\delta_{lk}\\
		\int_{-1}^1\frac1{1-x^2}P_l^m(x)P_l^n(x)\,\dd x=
		\begin{cases}
			0						&m\ne n\\
			\frac{(l+m)!}{m(l-m)!}	&m=n\ne 0\\
			+\infty					&m=n=0
		\end{cases}\\
		\int_{-1}^1P_l^m(x)P_k^n(x)\,\dd x=\frac{2(l+m)!}{(2l+1)(l-m)!}\delta_{lk}.
	\end{gathered}
	\label{eq:ortogonalita-legendre}
\end{equation}

