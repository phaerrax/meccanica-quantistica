\chapter{Sistemi unidimensionali}
In queto capitolo affronteremo una serie di esempi ``accademici'' di sistemi in una dimensione, cercando di risolvere dove possibile l'equazione di Schr\"odinger per la funzione d'onda e lo spettro di energia, o di trovare la migliore approssimazione sfruttando le tecniche acquisite.

\section{Particella libera}
Contrariamente al caso classico, in cui questo era il sistema più semplice da studiare, nel caso quantistico troviamo qualche complicazione.
L'hamiltoniano del sistema è semplicemente
\begin{equation}
	\op H=\frac1{2m}\op p^2.
	\label{eq:H-particella-libera}
\end{equation}
Notiamo immediatamente che $[\op p,\op H]=0$ essendo $\op H$ funzione unicamente di $\op p$, dunque le due osservabili sono compatibili.
L'impulso dunque è una quantità conservata: a questo corrisponde la simmetria (evidente) del sistema per traslazioni spaziali, che sono generate da $\op p$.
Preso un autostato $\ket{p}$ dell'impulso (di autovalore $p$), troviamo $\op H\ket{p}=\frac1{2m}\op p^2\ket{p}=\frac1{2m}p^2\ket{p}$.
D'altro canto, $\frac1{2m}p^2$ è l'autovalore dell'hamiltoniano dunque è l'energia $E$ del sistema; se fissiamo dunque questo valore, troviamo due autostati di $\op p$ aventi questa energia, che sono $\ket{p}$ e $\ket{-p}$.
Ciò significa che l'hamiltoniano è degenere, ossia un suo autostato $\ket{E}$ si scrive come $\alpha\ket{p}+\beta\ket{-p}$.

Nella base degli autostati della posizione, la funzione d'onda è soluzione dell'equazione di Schr\"odinger
\begin{equation}
	\frac{\hbar^2}{2m}\psi''(x)=E\psi(x)
	\label{eq:schrodinger-particella-libera}
\end{equation}
da cui
\begin{equation}
	\psi(x)=Ae^{-\frac{i}{\hbar}\sqrt{2mE}x}+Be^{\frac{i}{\hbar}\sqrt{2mE}x}
	\label{eq:wf-particella-libera}
\end{equation}
per qualche $A,B\in\C$.
L'energia $E$ è necessariamente positiva in quanto, come visto prima, è uguale a $\frac{p^2}{2m}$.
Alternativamente, se $E$ fosse negativa allora $\sqrt{E}$ avrebbe anche una parte immaginaria, che moltiplicata per $\pm i$ negli esponenti porterebbe a un'espressione della forma $e^{\pm\lambda x}$: in tal caso la funzione d'onda divergerebbe per $\abs{x}\to+\infty$ e non sarebbe accettabile (fisicamente) come soluzione.
Oltre a questo, $E$ può assumere qualsiasi valore reale positivo; in ogni caso, per nessun valore risulta $\psi\in L^2(\R)$, dato che la soluzione è oscillante.
Questo fatto non ci deve turbare, perch\'e sappiamo che l'impulso di uno stato non può essere conosciuto con assoluta precisione, mentre all'inizio del problema abbiamo preso proprio un autostato di $\op p$.
Dopotutto, se il sistema fosse nell'autostato $\ket{\pm p}$, avrebbe un'indeterminazione \emph{nulla} sull'impulso e di conseguenza, per il principio di Heisenberg, l'indeterminazione sulla posizione dovrà essere \emph{infinita} (ed è questo il caso) affinch\'e il prodotto $\Delta q\Delta p$ possa essere finito.

\section{Potenziale lineare}
Consideriamo il sistema formato da una particella soggetta al potenziale $V(x)=-ax$.
L'hamiltoniano del sistema è l'operatore
\begin{equation}
	\op H=\frac1{2m}\op p^2-a\op q.
	\label{eq:H-potenziale-lineare}
\end{equation}
Possiamo ricavare qualitativamente alcune informazioni sul sistema guardando al potenziale:
\begin{itemize}
	\item dato che $V\to-\infty$ per $x\to+\infty$, non ammette un minimo, perciò lo spettro di $\op H$ non potrà essere discreto e non possono esistere, di conseguenza, stati legati;
	\item d'altro canto $V\to+\infty$ per $x\to-\infty$ dunque $\op H$ non sarà degenere.
\end{itemize}
Nella base della posizione l'equazione di Schr\"odinger è
\begin{equation}
	-\frac{\hbar^2}{2m}\psi''(x)-(ax+E)\psi(x)=0
	\label{eq:schrodinger-posizione-potenziale-lineare}
\end{equation}
ossia
\begin{equation}
	\psi''(x)+\frac{2m}{\hbar^2}(E+ax)\psi(x)=0.
\end{equation}
Osserviamo che $\frac{2ma}{\hbar^2}$ ha le dimensioni di una $\textup{lunghezza}^{-3}$, e che possiamo raggruppare $a$ nell'equazione nel termine $(E+ax)\psi(x)$.
La variabile $x+\frac{E}{a}$ ha dunque le dimenzioni di una lunghezza: nell'equazione di Schr\"odinger operiamo dunque il cambio di variabile
\begin{equation}
	\xi=\bigg(\frac{2ma}{\hbar^2}\bigg)^{\frac13}\bigg(x-\frac{E}{a}\bigg).
\end{equation}
Per la derivata, abbiamo
\begin{equation}
	\drv{}{x}=\drv{}{\xi}\drv{\xi}{x}=\bigg(\frac{2ma}{\hbar^2}\bigg)^{\frac13}\drv{}{\xi}
\end{equation}
ottenendo la nuova equazione
\begin{equation}
	\begin{gathered}
		\bigg(\frac{2ma}{\hbar^2}\bigg)^{\frac23}\psi''(\xi)+\frac{2ma}{\hbar^2}\bigg(\frac{2ma}{\hbar^2}\bigg)^{-\frac13}\xi\psi(\xi)=0\\
		\bigg(\frac{2ma}{\hbar^2}\bigg)^{\frac23}\psi''(\xi)+\bigg(\frac{2ma}{\hbar^2}\bigg)^{\frac23}\xi\psi(\xi)=0\\
		\psi''(\xi)+\xi\psi(\xi)=0.
	\end{gathered}
	\label{eq:soluzione-potenziale-lineare}
\end{equation}
Cambiamo ancora variabile con $\zeta=-\xi$ (per cui si ha $\ddrv{}{\xi}=\ddrv{}{\zeta}$) per ottenere l'\emph{equazione di Airy}
\begin{equation}
	\psi''(\zeta)+\zeta\psi(\zeta)=0
	\label{eq:airy-potenziale-lineare}
\end{equation}
le cui due soluzioni indipendenti sono le omonime \emph{funzioni di Airy} del primo e del secondo tipo, denominate rispettivamente $\Ai\zeta$ e $\Bi\zeta$.
Sono funzioni particolari, non esprimibili solamente in termini di funzioni elementari.
In particolare, hanno un comportamento oscillatorio per $\zeta<0$ ed esponenziale per $\zeta>0$.
Dato che $\Bi\to+\infty$ esponenzialmente per $\zeta\to+\infty$, ci interessiamo d'ora in poi solo della funzione del primo tipo, $\Ai$: la funzione d'onda soluzione di \eqref{eq:airy-potenziale-lineare} ha dunque la forma $\psi(\zeta)=c\Ai(\zeta)$, o tornando nella variabile $\xi$ precedente $\psi(\xi)=c\Ai(-\xi)$, per qualche $c\in\C$.

Cerchiamo ora il comportamento asintotico di $\psi$ per valori molto grandi di $\xi$: ipotizzando che $\psi(\xi)\sim\exp(-\gamma{\xi}^s)$, con $\gamma,s>0$, troviamo
\begin{gather*}
	\psi'(\xi)=-(\sgn\xi)\gamma s\abs{\xi}^{s-1}e^{-\gamma\abs{\xi}^s}\\
	\psi''(\xi)=\big[s^2\gamma^2\abs{\xi}^{2s-2}-(\sgn\xi)\gamma s(s-1)\abs{\xi}^{s-2}\big]e^{-\gamma\abs{\xi}^s}.
\end{gather*}
Sostituendole nella \eqref{eq:soluzione-potenziale-lineare} otteniamo che deve essere $\abs{\xi}^{2s-2}\sim\abs{\xi}$ e $s^2\gamma^2=1$ affinch\'e la soluzione sia accettabile, perciò troviamo $s=\frac32$ e $\gamma=\frac23$.
La funzione d'onda approssimata per grandi valori di $\xi$ è dunque
\begin{equation}
	\psi(\xi)\sim ce^{-\frac23\abs{\xi}^{3/2}}
\end{equation}
con $c$ da determinare normalizzando.
Otteniamo una soluzione più ``fine'' ipotizzando che $\psi(\xi)\sim c\exp(-\frac23\abs{\xi}^{3/2})\abs{\xi}^\beta$ per la quale si ottiene $\beta=-\frac14$.

Sebbene sia più ``naturale'' lavorare nella base della posizione, in questo caso risulta più comodo usare la base dell'impulso, perch\'e non appaiono potenze maggiori di $\op q$ nell'hamiltoniano: si ottiene dunque un'equazione differenziale del primo, e non del secondo, ordine.
L'equazione di Schr\"odinger (chiamiamo $\tilde{\psi}$ la funzione d'onda nello spazio degli impulsi per evitare confusioni) in questa base è dunque
\begin{equation}
	\frac1{2m}p^2\tilde{\psi}(p)-ia\hbar\tilde{\psi}'(p)=E\tilde{\psi}(p)
	\label{eq:schrodinger-impulso-potenziale-lineare}
\end{equation}
da cui
\begin{equation}
	\tilde{\psi}'(p)=\frac{i}{a\hbar}\bigg(E-\frac{p^2}{2m}\bigg)\tilde{\psi}(p)
\end{equation}
che ha come soluzione la funzione d'onda
\begin{equation}
	\tilde{\psi}(p)=Ae^{\frac{i}{a\hbar}\big(Ep-\frac{p^3}{6m}\big)}.
\end{equation}

Torniamo dunque allo spazio della posizione con la trasformata di Fourier:
\begin{equation}
	\begin{split}
		\psi(x)&=(\four{\tilde{\psi}})(x)=\frac{A}{\sqrt{2\pi\hbar}}\int_{-\infty}^{+\infty}e^{-\frac{i}{\hbar}px}\tilde{\psi}(p)\,\dd p=\\
		&=\frac{A}{\sqrt{2\pi\hbar}}\int_{-\infty}^{+\infty}\exp\bigg[\frac{i}{\hbar}p\bigg(x+\frac{E}{a}\bigg)+\frac{ip^3}{6am\hbar}\bigg]\,\dd p=\\
		&=\frac{A}{\sqrt{2\pi\hbar}}\int_{-\infty}^{+\infty}\exp\bigg[\frac{i}{\hbar}p\xi\bigg(\frac{2am}{\hbar^2}\bigg)^{-\frac13}+\frac{ip^3}{6am\hbar}\bigg]\,\dd p=\\ &=\tilde{A}\int_{-\infty}^{+\infty}\exp\bigg(ip'\xi+\frac{ip'^3}3\bigg)\,\dd p'
	\end{split}
\end{equation}
ponendo $p'=\frac{p}{\hbar}\big(\frac{2am}{\hbar^2}\big)^{-\frac13}$.
Il risultato è un'espressione integrale proprio della funzione $\Ai(\xi)$, dunque (dopo una normalizzazione per determinare il valore di $\tilde{A}$) si ottiene $\psi(\xi)$.

