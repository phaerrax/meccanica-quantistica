\chapter{Evoluzione temporale}
Poich\'e finora ci siamo interessati degli stati di equilibrio dei sistemi  e delle loro energie, ci è servito studiare soltanto una descrizione indipendente dal tempo.
Per analizzare i fenomeni dinamici e variabili nel tempo, però, ci occorre capire come gli stati dei sistemi quantistici evolvono nel tempo: dato uno stato $\ket{\psi(0)}$ noto ad un certo istante iniziale di tempo, come determinare la sua evoluzione $\ket{\psi(t)}$ ad un generico istante?
Nella meccanica classica questo fenomeno è descritto dall'evoluzione delle coordinate canoniche nello spazio delle fasi; in meccanica quantistica, esso è descritto da un \emph{operatore di propagazione} (o soltanto \emph{propagatore}), sia esso $\op U(t)$, per il quale $\ket{\psi(t)}=\op U(t)\ket{\psi(0)}$.
Seguendo l'esempio della meccanica classica, assumiamo anche qui che l'hamiltoniano sia il generatore delle traslazioni temporali infinitesime.
Se l'hamiltoniano stesso non dipende dal tempo, scriviamo dunque
\begin{equation}
	\op U(t)=e^{-\frac{i}{\hbar}\op Ht}.
	\label{eq:propagatore}
\end{equation}
Notiamo come con questa forma il propagatore rispetta, in modo ovvio, le proprietà che ci aspettiamo possieda:
\begin{itemize}
	\item conserva la probabilità: $\braket{\psi(t)}{\psi(t)}=\bra{\psi(0)}\adj{\op U}(t)\op U(t)\ket{\psi(0)}=\braket{\psi(0)}{\psi(0)}$;
	\item è lineare e continuo, quindi $U(t)\to\op 1$ per $t\to 0$;
	\item si compone con la regola $\op U(t_1)\op U(t_2)=\op U(t_1+t_2)$.
\end{itemize}
In altre parole, questi operatori formano un gruppo unitario continuo ad un parametro.

Per studiare come uno stato varia nel tempo consideriamo la sua derivata temporale all'istante $t$, passando alla derivata di $\op U(t)$, possiamo ottenere l'equazione differenziale
\begin{equation}
	\drp{}{t}\ket{\psi(t)}=\drp{}{t}\op U(t)\ket{\psi(0)}=-\frac{i}{\hbar}\op H\op U(t)\ket{\psi(0)}=-\frac{i}{\hbar}\op H\ket{\psi(t)}\quad\longrightarrow\quad \op H=i\hbar\drp{}{t}.
\end{equation}
Nel caso in cui anche l'hamiltoniano dipenda dal tempo, generalizziamo l'equazione nella forma
\begin{equation}
	i\hbar\drp{}{t}\op U(t)=\op H(t)\op U(t)
	\label{eq:schrodinger-tempo}
\end{equation}
detta \emph{equazione di Schr\"odinger dipendente dal tempo}.
Nella rappresentazione delle coordinate troviamo l'equazione differenziale per la funzione d'onda, che è $\bra{q}i\hbar\drp{}{t}\ket{\psi(t)}=\bra{q}\op H(t)\ket{\psi(t)}$, da cui
\begin{equation}
	i\hbar\drp{}{t}\psi(\vec q,t)=H\psi(\vec q,t).
	\label{eq:schrodinger-tempo-coordinate}
\end{equation}

\section{Stati stazionari}
Esistono degli stati \emph{stazionari}, che non dipendono dal tempo?
Chiedere che $\ket{\psi(t)}$ sia uguale a $\ket{\psi(0)}$ è eccessivo, perch\'e allora $\ket{\psi(0)}$ sarebbe un autostato di $\op U(t)$ con autovalore 1, ossia
\begin{equation}
	\ket{\psi(0)}=e^{-\frac{i}{\hbar}\op Ht}\ket{\psi(0)}=\sum_{k=0}^{+\infty}\frac1{k!}\bigg(-\frac{i}{\hbar}\op Ht\bigg)^k\ket{\psi(0)}=\ket{\psi(0)}+\sum_{k=1}^{+\infty}\frac1{k!}\bigg(-\frac{i}{\hbar}\op Ht\bigg)^k\ket{\psi(0)}
\end{equation}
che vale a dire che $\ket{\psi(0)}$ è un autostato di $\op H$ con autovalore nullo, situazione impossibile per uno stato fisico.
Sappiamo però che due vettori multipli per uno scalare rappresentano lo stesso stato fisico, perciò anche se $\op U(t)\ket{\psi(0)}=c(t)\ket{\psi(0)}$ lo stato fisico rimane invariato, come vogliamo.
Oltre ad apparire nell'espressione del propagatore, l'hamiltoniano gioca un ruolo cruciale nell'evoluzione temporale, come nel seguente teorema.
\begin{teorema} \label{t:autostati-hamiltoniano-stazionari}
	Uno stato è stazionario se e solo se è un autostato dell'hamiltoniano.
\end{teorema}
\begin{proof}
	Sia $\ket{\psi(0)}$ un autostato di $\op H$: allora $\op H\ket{\psi(0)}=E\ket{\psi(0)}$ per un certo $E\in\R$.
	Sviluppando in serie l'esponenziale, ogni termine ha una potenza di $\op H$ e chiaramente $\op H^k\ket{\psi(0)}=E^k\ket{\psi(0)}$, quindi $\op U(t)\ket{\psi(0)}=\exp\big(-\frac{i}{\hbar}\op Ht\big)\ket{\psi(0)}=\exp\big(-\frac{i}{\hbar}Et\big)\ket{\psi(0)}$.

	Sia ora $\ket{\psi(0)}$ uno stato stazionario, per cui esiste una funzione $c(t)$ tale che $\ket{\psi(t)}=c(t)\ket{\psi(0)}$.
	Allora dall'equazione \eqref{eq:schrodinger-tempo} applicata a $\ket{\psi(0)}$ troviamo
	\begin{equation}
		i\hbar\dot{c}(t)\ket{\psi(0)}=c(t)\op H\ket{\psi(0)}\quad\longrightarrow\quad i\hbar\frac{\dot{c}(t)}{c(t)}\ket{\psi(0)}=\op H\ket{\psi(0)}.
	\end{equation}
	Ora, se $\op H$ non dipende dal tempo, anche il primo membro è una costante.
	Poniamo dunque porre $E\defeq i\hbar\frac{\dot{c}(t)}{c(t)}$ ottenendo l'equazione differenziale $\dot{c}(t)=-\frac{i}{\hbar}Ec(t)$.
	La condizione iniziale è $c(0)=1$, dato che deve valere evidentemente $\ket{\psi(0)}=c(0)\ket{\psi(0)}$.
	Troviamo di conseguenza la soluzione $c(t)=\exp\big(-\frac{i}{\hbar}Et\big)$, vale a dire $\ket{\psi(0)}$ è un autostato di $\op H$ come si può vedere ricalcando il ragionamento della prima parte della dimostrazione.
\end{proof}

Conoscendo lo spettro dell'hamiltoniano e i suoi autostati $\ket{E_n}$, potremmo sviluppare lo stato in questa base scrivendo
\begin{equation}
	\ket{\psi(0)}=\sum_{n=0}^{+\infty}\ket{E_n}\braket{E_n}{\psi(0)}=\sum_{n=0}^{+\infty}a_n\ket{E_n}
\end{equation}
da cui ricaviamo
\begin{equation}
	\ket{\psi(t)}=\op U(t)\sum_{n=0}^{+\infty}a_n\ket{E_n}=\sum_{n=0}^{+\infty}a_n\op U(t)\ket{E_n}=\sum_{n=0}^{+\infty}a_ne^{-\frac{i}{\hbar}E_nt}\ket{E_n},
\end{equation}
quindi i coefficienti dello sviluppo nella base di autostati di $\op H$ evolvono secondo l'equazione $a_n(t)=a_n(0)\exp\big(-\frac{i}{\hbar}E_nt\big)$, poich\'e gli autostati sono stazionari.

\section{Evoluzione delle osservabili}
Finora abbiamo visto gli stati variare nel tempo, mentre le osservabili rimangono costanti: è lo \emph{schema di Schr\"odinger}.
Potremmo invece lasciare fissati gli stati e far variare, secondo le medesime equazioni, le osservabili ( \emph{schema di Heisenberg}).
Possiamo individuare delle osservabili i cui valori medi rimangono costanti?
Il valore medio, come funzione del tempo, è ancora definito come $\avg{\xi(t)}=\bra{\psi(t)}\op\xi\ket{\psi(t)}$.
Calcolando la sua derivata rispetto al tempo e uguagliandola a zero troviamo, dato che $\bra{\psi(0)}$ e $\ket{\psi(0)}$ sono costanti,
\begin{multline}
	0=\drp{}{t}[\adj{\op U}(t)\op\xi\op U(t)]=\bigg(\drp{}{t}\adj{\op U}(t)\bigg)\op\xi\op U(t)+\adj{\op U}(t)\op\xi\drp{}{t}\op U(t)=\\
	=\frac{i}{\hbar}\big[\adj{\op U}(t)\op H\op\xi\op U(t)-\adj{\op U}(t)\op\xi\op H\op U(t)\big]=\frac{i}{\hbar}\adj{\op U}(t)[\op H,\op\xi]\op U(t)
\end{multline}
ossia $[\op H,\op\xi]=0$.
Dunque $\op\xi$ è una ``costante del moto'' se commuta con $\op H$: in questo caso sono compatibili (se $\op H$ non è degenere) e possiamo scegliere una base di autostati simultanei, che sono quindi autostati di $\op\xi$ stazionari.

Nell'equazione $\bra{\psi(t)}\op\xi\ket{\psi(t)}=\bra{\psi(0)}\adj{\op U}(t)\op\xi\op U(t)\ket{\psi(0)}$ possiamo definire l'operatore variabile nel tempo $\op\xi(t)\defeq\adj{\op U}(t)\op\xi\op U(t)$ ottenendo $\bra{\psi(t)}\op\xi\ket{\psi(t)}=\bra{\psi(0)}\op\xi(t)\ket{\psi(0)}$.
Se $\op\xi$ è hermitiano chiaramente lo è anche $\op\xi(t)$, inoltre vale ancora
\begin{equation}
	\drp{}{t}\op\xi(t)=\frac{i}{\hbar}\adj{\op U}(t)[\op H,\op\xi]\op U(t)=\frac{i}{\hbar}[\op H,\op\xi(t)],
	\label{eq:evoluzione-osservabili-heisenberg}
\end{equation}
detta qui \emph{equazione di Heisenberg}.
A meno del fattore $i/\hbar$, questa equazione non è altro che la già nota relazione in meccanica classica per l'evoluzione temporale delle osservabili $\dot{f}=\{f,H\}$.

Ad esempio, prendiamo l'hamiltoniano $\op H=\frac1{2m}\op p^2+V(\op q)$ di una particella soggetta, in una dimensione, ad un generico potenziale $V(q)$.
L'evoluzione della posizione è
\begin{equation}
	\drp{}{t}\op q(t)=\frac{i}{\hbar}[\op H,\op q]=\frac{i}{2\hbar m}[\op p^2,\op q]=\frac{i\op p}{\hbar m}[\op p,\op q]=\frac{\op p}{m},
\end{equation}
mentre quella degli impulsi è
\begin{equation}
	\drp{}{t}\op p(t)=\frac{i}{\hbar}[\op H,\op p]=\frac{i}{\hbar}[V(\op q),\op p]=-\drp{V}{q}(\op q).
\end{equation}
Notare la forte somiglianza di queste due equazioni con quelle di Hamilton in meccanica classica.
