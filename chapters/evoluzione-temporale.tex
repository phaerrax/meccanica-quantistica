\chapter{Evoluzione temporale}
Poich\'e finora ci siamo interessati degli stati di equilibrio dei sistemi  e delle loro energie, ci è servito studiare soltanto una descrizione indipendente dal tempo.
Per analizzare i fenomeni dinamici e variabili nel tempo, però, ci occorre capire come gli stati dei sistemi quantistici evolvono nel tempo: dato uno stato $\ket{\psi(0)}$ noto ad un certo istante iniziale di tempo, come determinare la sua evoluzione $\ket{\psi(t)}$ ad un generico istante?
Nella meccanica classica questo fenomeno è descritto dall'evoluzione delle coordinate canoniche nello spazio delle fasi; in meccanica quantistica, esso è descritto da un \emph{operatore di propagazione} (o soltanto \emph{propagatore}), sia esso $\op U(t)$, per il quale $\ket{\psi(t)}=\op U(t)\ket{\psi(0)}$.
Seguendo l'esempio della meccanica classica, assumiamo anche qui che l'hamiltoniano sia il generatore delle traslazioni temporali infinitesime.
Se l'hamiltoniano stesso non dipende dal tempo, scriviamo dunque
\begin{equation}
	\op U(t)=e^{-\frac{i}{\hbar}\op Ht}.
	\label{eq:propagatore}
\end{equation}
Notiamo come con questa forma il propagatore rispetta, in modo ovvio, le proprietà che ci aspettiamo possieda:
\begin{itemize}
	\item conserva la probabilità: $\braket{\psi(t)}{\psi(t)}=\bra{\psi(0)}\adj{\op U}(t)\op U(t)\ket{\psi(0)}=\braket{\psi(0)}{\psi(0)}$;
	\item è lineare e continuo, quindi $U(t)\to\op 1$ per $t\to 0$;
	\item si compone con la regola $\op U(t_1)\op U(t_2)=\op U(t_1+t_2)$.
\end{itemize}
In altre parole, qusti operatori formano un gruppo unitario continuo ad un parametro.

Dalla derivata temporale dello stato all'istante $t$, passando alla derivata di $\op U(t)$, possiamo ottenere l'equazione differenziale
\begin{equation}
	\drp{}{t}\ket{\psi(t)}=\drp{}{t}\op U(t)\ket{\psi(0)}=-\frac{i}{\hbar}\op H\op U(t)\ket{\psi(0)}=-\frac{i}{\hbar}\op H\ket{\psi(t)}\quad\longrightarrow\quad \op H=i\hbar\drp{}{t}.
\end{equation}
Nel caso in cui anche l'hamiltoniano dipenda dal tempo, generalizziamo l'equazione nella forma
\begin{equation}
	i\hbar\drp{}{t}\op U(t)=\op H(t)\op U(t)
	\label{eq:schrodinger-tempo}
\end{equation}
detta \emph{equazione di Schr\"odinger dipendente dal tempo}.
Nella rappresentazione delle coordinate troviamo l'equazione differenziale per la funzione d'onda, che è $\bra{q}i\hbar\drp{}{t}\ket{\psi(t)}=\bra{q}\op H(t)\ket{\psi(t)}$ da cui
\begin{equation}
	i\hbar\drp{}{t}\psi(\vec q,t)=H\psi(\vec q,t).
	\label{eq:schrodinger-tempo-coordinate}
\end{equation}
